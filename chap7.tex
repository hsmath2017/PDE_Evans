\chapter{Linear Evolution Equations}
In this chapter, we will study the linear PDEs that involves time. We call such PDE \textbf{evolution equations}. Now, we study the \textbf{general second-order parabolic and hyperbolic equations} first. In this chapter, \textbf{the Fourier transform} and \textbf{the semigroup techniques} provide alternative approaches.
\section{The space $H^{-1}(U)$}
In this short section, we will introduce the definition and some basic properties of the Sobolev space $H^{-1}(U)$. 
\begin{definition}
    We denote by $H^{-1}(U)$ the dual space to $H_{0}^{1}(U)$.
\end{definition}
In other words, $f$ belongs to $H^{-1}(U)$ provided $f$ is a bounded linear functional on $H_{0}^{1}(U)$. In fact, $H_{0}^{1}(U)$ isn't reflexive, and we have:
\begin{equation}
    H_{0}^{1}(U)\subset L^{2}(U)\subset H^{-1}(U).
\end{equation}
We write $\innerprod{\cdot}{\cdot}$ to denote the pairing between $H^{-1}(U)$ and $H_{0}^{1}(U)$.
\begin{definition}
    If $f\in H^{-1}(U)$, we define the norm 
    \begin{equation}
        \norm{f}_{H^{-1}(U)}:=\sup\left\{\innerprod{f}{u}|u\in H_{0}^{1}(U),\norm{u}_{H_{0}^{1}(U)}\le 1\right\}.
    \end{equation}
\end{definition}
Then, we will introduce the characterizations of $H^{-1}(U)$.
\begin{theorem}[characterization of $H^{-1}$]
\begin{itemize}
    \item Assume $f\in H^{-1}(U)$, there exist functions $f^{0},f^{1},\cdots,f^{n}$ in $L^{2}(U)$ such that 
    \begin{equation}
        \innerprod{f}{v}=\int_{U}f^{0}v+\sum_{i=1}^{n}f^{i}v_{x_{i}}\dif x.
    \end{equation}
    \item Furthermore,
    \begin{equation}
        \norm{f}_{H^{-1}(U)}=\inf\left\{\left(\int_{U}\sum_{i=0}^{n}|f^{i}|^{2}\dif x\right)^{\frac{1}{2}}\right\}.
    \end{equation}
    \item In particular, we have
    \begin{equation}
        (v,u)_{L^{2}(U)}=\innerprod{v}{u}
    \end{equation}
    for all $u\in H_{0}^{1}(U)$, $v\in L^{2}(U)\subset H^{-1}(U)$.
\end{itemize}
\end{theorem}
\section{Second-order Parabolic Equations}
Second-order parabolic PDEs are natural generalizations of the heat equations. We will study the existence and uniqueness of appropriately defined weak solutions, their smoothness and other properties.
\subsection{Definitions}
\subsubsection{Parabolic equations}
In this chapter, we assume $U$ to be an open bounded subset of $\mathbb{R}^{n}$ and $U_{T}:=U\times(0,T]$ for some fixed time $T>0$.

Consider the initial/boundary-value problem:
\begin{equation}
    \label{eq:Parabolic_IBVP}
    \left\{
        \begin{aligned}
            u_{t}+Lu&=f,x\in U_{T};\\
            u&=0,x\in \partial U\times[0,T];\\
            u&=g,x\in U\times\{t=0\}.
        \end{aligned}
    \right.
\end{equation}
where $f:U_{T}\rightarrow\mathbb{R}$ and $g:U\rightarrow\mathbb{R}$ are given and $u:\bar{U_{T}}\rightarrow\mathbb{R}$ is the unknown. The operator $L$ is the elliptic operator related to time parameter $t$ as divergence form \eqref{eq:div-form_ellip} or non-divergence form \eqref{eq:non_div-form_ellip}, with the uniformly parabolic condition.
\begin{definition}[Uniformly parabolic]
    We say that the partial differential operator $\pdfFrac{}{t}+L$ is uniformly parabolic if there exists a constant $\theta>0$ such that 
    \begin{equation}
        \sum_{i,j=1}^{n}a^{ij}(x,t)\xi_{i}\xi_{j}\ge \theta|\xi|^{2}
    \end{equation}
    for all $(x,t)\in U_{T}$, $\xi\in\mathbb{R}^{n}$.
\end{definition}
In physical terms, the second-order term $\sum a^{ij}u_{i}u_{j}$ represents \textbf{diffusion}, the first-order term $\sum b^{i}u_{i}$ represents \textbf{transport}, and the zeroth-order term $cu$ represents \textbf{creation}.
\subsubsection{Weak solutions}
Now, we consider the operator $L$ has the divergence form , and:
\begin{equation}
    \label{eq:weak_sol_conditions}
    \begin{aligned}
        &a^{ij},b^{i},c\in L^{\infty}(U_{T}),\\
        &f\in L^{2}(U_{T}),\\
        &g\in L^{2}(U).\\
    \end{aligned}
\end{equation}
We will always suppose $a^{ij}=a^{ji}$.

First, we temporarily suppose that $u(x,t)$ is a smooth function on $U_{T}$. Define a series of function $\mathbf{u}:[0,T]\rightarrow H_{0}^{1}(U)$ as 
\begin{equation}
    [\mathbf{u}(t)](x)=u(x,t).
\end{equation}
and a series $\mathbf{f}:[0,T]\rightarrow L^{2}(U)$ as:
\begin{equation}
    [\mathbf{f}(t)](x):=f(x,t).
\end{equation}
Now, choose a fixed function $v\in H_{0}^{1}(U)$, multiply the PDE $\pdfFrac{u}{t}+Lu=f$ by $v$ and integrate by parts, define the bilinear form 
\begin{equation}
    \label{eq:bilinear_for_parabolic}
    B[u,v;t]:=\int_{U}\left(\sum_{i,j=1}^{n}a^{ij}(\cdot,t)u_{i}v_{j}+\sum_{i=1}^{n}b^{i}(\cdot,t)u_{i}v+cuv\right)\dif x,
\end{equation}
we can see that 
\begin{equation}
    \label{eq:variation_form}
    (\mathbf{u}',v)+B[\mathbf{u},v;t]=(\mathbf{f},v)
\end{equation}
\textbf{for each }$0\le t\le T$. The pairing $(\cdot,\cdot)$ denoting the inner product in $L^{2}(U)$.

Then, by \eqref{eq:Parabolic_IBVP}, we can see:
\begin{equation}
    u_{t}=g^{0}+\sum_{j=1}^{n}g_{x_{j}}^{j},x\in U_{T}
\end{equation}
for $g^{0}:=f-\sum_{i=1}^{n}b^{i}u_{x_{i}}-cu$ and $g^{j}:=\sum_{i=1}^{n}a^{ij}u_{x_{i}}$. By the definition of $\norm{\cdot}_{H^{-1}(U)}$, we can see:
\begin{equation}
    \norm{u_{t}}_{H^{-1}(U)}\le\left(\sum_{j=0}^{n}\norm{g^{j}}_{L^{2}(U)}^{2}\right)^{\frac{1}{2}}\le C\left(\norm{u}_{H_{0}^{1}(U)}+\norm{f}_{L^{2}(U)}\right).
\end{equation}
So, it's reasonable to demand $u_{t}\in H^{-1}(U)$. In the case where $u$ smooth on $U_{T}$, we observe that $u_{t}(\tau)\in L^{2}(U)$. Consequently, the term $(\mathbf{u}',v)$ in equation \eqref{eq:variation_form} can be generally expressed as $\innerprod{\mathbf{u}'}{v}$ on $H_{0}^{1}(U)$. From this, we can derive the definition of a weak solution.
\begin{definition}[Weak Solution]
    \label{defn:weak_sol_for_parabolic}
    We say a function 
    \begin{equation}
        \mathbf{u}\in L^{2}(0,T;H_{0}^{1}(U)),\mathbf{u}'\in L^{2}(0,T;H^{-1}(U)),
    \end{equation}
    is a \textbf{weak solution} of the parabolic initial/boundary value problem \eqref{eq:Parabolic_IBVP} provided 
    \begin{equation}
        \label{eq:variation_func}
        \innerprod{\mathbf{u}'}{v}+B[\mathbf{u},v;t]=(\mathbf{f},v)
    \end{equation}
    for each $v\in H_{0}^{1}(U)$ and a.e. time $0\le t\le T$ and $\mathbf{u}(0)=g$.
\end{definition}
\begin{remark}
    The space $L^{2}(0,T;H_{0}^{1}(U))$ means that:
    \begin{itemize}
        \item $\forall t\in [0,T]$, $\mathbf{u}(t)\in H_{0}^{1}(U)$.
        \item Mark $f(t):=\norm{\mathbf{u}(t)}_{H_{0}^{1}(U)}$, then $f(t)\in L^{2}(0,T)$.
    \end{itemize}
\end{remark}
\subsection{Existence and Uniqueness}
In this section, we will discuss the \textbf{existence} and \textbf{uniqueness} of weak solutions. In Chapter 1, we explored the well-posedness of elliptic equations through the analysis of compact operators. Here, our focus shifts to evolution equations. As the parameter $t$ varies, we select finite-dimensional subspaces $V_{i}$ to approximate the functional space $H_{0}^{1}(U)$. We then seek the approximated solution on $V_{i}$ and take the limit, employing the method known as \textbf{Galerkin approximations}.
\subsubsection{Galerkin approximations}
For the parabolic problem \eqref{eq:Parabolic_IBVP}, we construct the solutions of certain finite-dimensional approximations first. More precisely, by theorem \ref{thm:eig_val_of_sym_op}, if we choose the set of eigenfunctions $\{w_{k}(x)\}$ for the operator $L=-\Delta$, then $\{w_{k}\}_{k=1}^{\infty}$ forms an orthogonal basis of $H_{0}^{1}(U)$, and $\{w_{k}\}_{k=1}^{\infty}$ forms an orthonormal basis of $L^{2}(U)$.

Now, fix $m\in\mathbb{N}^{+}$, we will look for a function $\mathbf{u}_{m}:[0,T]\rightarrow H_{0}^{1}(U)$ with the form:
\begin{equation}
    \label{eq:finite_dim_approx}
    \mathbf{u}_{m}(t):=\sum_{k=1}^{m}d_{m}^{k}(t)w_{k},
\end{equation}
by the definition \ref{defn:weak_sol_for_parabolic}, the finite dimensional approximation $d_{m}^{k}(t)$ satisfies:
\begin{itemize}
    \item 
    \begin{equation}
        \label{eq:bdry_value}
        d_{m}^{k}(0)=(g,w_{k}).
    \end{equation}
    \item 
    \begin{equation}
        \label{eq:main_function}
        (\mathbf{u}'_{m},w_{k})+B[\mathbf{u}_{m},w_{k};t]=(\mathbf{f},w_{k}).    
    \end{equation}
\end{itemize}
In fact, equation \eqref{eq:bdry_value} is obtained from the boundary value $\mathbf{u}(0) = g$. On the subspace $V_{k}$, we can observe that $(\mathbf{u}_{m}(0),w_{k}) = d_{m}^{k}(0) = (g, w_{k})$. Equation \eqref{eq:main_function} is derived from \eqref{eq:variation_func} by replacing $v$ with $w_{k}$. Thus, the function $\mathbf{u}_{m}(t)$ is the \textbf{projection} of the solution $\mathbf{u}$ onto the subspace $\text{span}\{w_{1},\cdots,w_{m}\}$.
\begin{theorem}[Construction of approximate solutions]
    For each integer $m=1,2,\cdots$ there exists a unique function $\mathbf{u}_{m}$ of the form \eqref{eq:finite_dim_approx} satisfying \eqref{eq:bdry_value} and \eqref{eq:main_function}.
\end{theorem}
\begin{proof}
    By the definition of $\mathbf{u}_{m}$, we can see:
    \begin{equation}
        \label{eq:ode_left}
        (\mathbf{u}'_{m},w_{k})=d_{m}^{k\;'}(t)
    \end{equation}
    and 
    \begin{equation}
        \label{eq:ode_right}
        B[\mathbf{u}_{m},w_{k};t]=\sum_{l=1}^{m}B[w_{l},w_{k};t]d_{m}^{l}(t).
    \end{equation}
    Now we get an ODE system:
    \begin{equation}
        \label{eq:ode}
        d_{m}^{k\;'}(t)=\sum_{l=1}^{m}B[w_{l},w_{k};t]d_{m}^{l}(t),
    \end{equation}
    with the initial values
    \begin{equation}
        \label{eq:ode_init_val}
        d_{m}^{k}(0)=(g,w_{k}).
    \end{equation}
    By the standard existence theory for ODEs, the equation \eqref{eq:ode} equipped with initial value \eqref{eq:ode_init_val} has unique solution $d_{m}^{k}(t)$. Then we can see the solution $\mathbf{u}_{m}$ exists and unique.
\end{proof}
\subsubsection{Energy estimates}
In this section, we need some uniform estimates for the $L^{2}$ norm of $\mathbf{u}_{m}$. It's just the \textbf{energy estimates} for parabolic equation.
\begin{theorem}[Energy estimates]
    There exists a constant $C$, depending only on $U,T$ and the coefficients of $L$, such that:
    \begin{equation}
        \label{eq:energy_estimate}
        \max_{0\le t\le T}\norm{\mathbf{u}_{m}(t)}_{L^{2}(U)}+\norm{\mathbf{u}_{m}}_{L^{2}(0,T;H_{0}^{1}(U))}+\norm{\mathbf{u}'_{m}}_{L^{2}(0,T;H^{-1}(U))}\le C\left(\norm{\mathbf{f}}_{L^{2}(0,T;L^{2}(U))}+\norm{g}_{L^{2}(U)}\right).
    \end{equation}
\end{theorem}
\begin{proof}
    First, by equation \eqref{eq:main_function}, we can see that 
    \begin{equation}
        \label{eq:variation_for_um}
        (\mathbf{u}'_{m},\mathbf{u}_{m})+B[\mathbf{u}_{m},\mathbf{u}_{m};t]=(\mathbf{f},\mathbf{u}_{m})
    \end{equation}
    for a.e. $0\le t\le T$. By the energy estimate for elliptic operator \eqref{eq:elliptic_B}, we can see:
    \begin{equation}
        \label{eq:parabolic_B}
        \beta\norm{\mathbf{u}_{m}}_{H_{0}^{1}(U)}^{2}\le B[\mathbf{u}_{m},\mathbf{u}_{m};t]+\gamma\norm{\mathbf{u}_{m}}_{L^{2}(U)}^{2}.
    \end{equation}
    So:
    \begin{equation}
        \frac{1}{2}\frac{\dif \norm{\mathbf{u}_{m}}_{L^{2}(U)}^{2}}{\dif t}+\beta\norm{\mathbf{u}_{m}}_{H_{0}^{1}(U)}^{2}\le\gamma\norm{\mathbf{u}_{m}}_{L^{2}(U)}^{2}+(\mathbf{f},\mathbf{u}_{m}).
    \end{equation}
    By Cauchy-Schwarz inequality, write $\eta(t):=\norm{\mathbf{u}_{m}(t)}_{L^{2}(U)}^{2}$, $\xi(t):=\norm{\mathbf{f}(t)}_{L^{2}(U)}^{2}$, we can see:
    \begin{equation}
        \eta'(t)\le C_{1}\eta(t)+C_{2}\xi(t).
    \end{equation}
    Thus, by Gronwall's inequality:
    \begin{equation}
        \eta(t)\le e^{C_{1}t}\left(\eta(0)+C_{2}\int_{0}^{t}\xi(\tau)\dif\tau\right).
    \end{equation}
    By the definition, $\eta(0)\le\norm{g}_{L^{2}(U)}^{2}$, i.e. 
    \begin{equation}
        \label{eq:estimate_um}
        \max_{0\le t\le T}\norm{\mathbf{u}_{m}}_{H_{0}^{1}(U)}^{2}\le C\left(\norm{g}_{L^{2}(U)}^{2}+\norm{\mathbf{f}}_{L^{2}(0,T;L^{2}(U))}^{2}\right).
    \end{equation}
    Then, consider $\norm{\mathbf{u}_{m}}_{L^{2}(0,T;H_{0}^{1}(U))}^{2}$, by equation \eqref{eq:estimate_um}, we can see:
    \begin{equation}
        \begin{aligned}
            \norm{\mathbf{u}_{m}}_{L^{2}(0,T;H_{0}^{1}(U))}^{2}&=\int_{0}^{T}\norm{\mathbf{u}_{m}(t)}_{H_{0}^{1}(U)}^{2}\dif t\\
            &\le \int_{0}^{T}C_{1}\norm{\mathbf{f}(t)}_{L^{2}(U)}^{2}+C_{2}\norm{\mathbf{u}_{m}(t)}_{L^{2}(U)}^{2}+C_{3}\frac{\dif \norm{\mathbf{u}_{m}}_{L^{2}(U)}^{2}}{\dif t}\dif t\\
            &\le C\left(\norm{g}_{L^{2}(U)}^{2}+\norm{\mathbf{f}}_{L^{2}(0,T;L^{2}(U))}^{2}\right).
        \end{aligned}
    \end{equation}
    Finally, fix any $v\in H_{0}^{1}(U)$ with $\norm{v}_{H_{0}^{1}(U)}\le 1$ and write $v=v^{1}+v^{2}$, where $v^{1}\in\text{span}\{w_{k}\}_{k=1}^{m}$ and $(v^{2},w_{k})=0$. Since $\{w_{k}\}_{k=0}^{\infty}$ are orthogonal in $H_{0}^{1}(U)$, and $\norm{v^{1}}_{H_{0}^{1}(U)}\le\norm{v}_{H_{0}^{1}(U)}\le 1$. By \eqref{eq:main_function}, for a.e. $0\le t\le T$, we have:
    \begin{equation}
        (\mathbf{u}'_{m},v^{1})+B[\mathbf{u}_{m},v^{1};t]=(\mathbf{f},v^{1}).
    \end{equation}
    Then:
    \begin{equation}
        \innerprod{\mathbf{u}'_{m}}{v}=(\mathbf{u}'_{m},v^{1})=\innerprod{\mathbf{f}}{v^{1}}-B[\mathbf{u}_{m},v^{1};t].
    \end{equation}
    Consequently:
    \begin{equation}
        \norm{\mathbf{u}'_{m}}_{H^{-1}(U)}\le C\left(\norm{f}_{L^{2}(U)}+\norm{\mathbf{u}_{m}}_{H_{0}^{1}(U)}\right),
    \end{equation}
    and therefore:
    \begin{equation}
        \int_{0}^{T}\norm{\mathbf{u}'_{m}}_{H^{-1}(U)}^{2}\le C\left(\norm{g}_{L^{2}(U)}^{2}+\norm{\mathbf{f}}_{L^{2}(0,T;L^{2}(U))}^{2}\right).
    \end{equation}
    Q.E.D.
\end{proof}