\chapter{Elliptic equation}
In this chapter, we will discuss on the general elliptic PDEs and its weak form. We will exploit two essentially distinct techniques, energy methods within Sobolev spaces and maximum principle methods.
\section{Definition}
\subsection{Elliptic equations}
We will in this chapter mostly study the boundary value problem(BVP):
\begin{equation}
    \label{eq:Elliptic_eq}
    \left\{
        \begin{aligned}
            Lu&=f\text{ in }U;\\
            u&=0\text{ on }\partial U,\\
        \end{aligned}
    \right.
\end{equation} 
where $U$ is an open, bounded subset of $\mathbb{R}^{n}$, and $u:\bar{U}\rightarrow\mathbb{R}$ is the unknown. Here $f$ is given, and $L$ denotes a second order differential operator have either the form:
\begin{equation}
    \label{eq:div-form_ellip}
    Lu=-\sum_{i,j=1}^{n}\left(a^{ij}(x)u_{x_{i}}\right)_{x_{j}}+\sum_{i=1}^{n}b^{i}(x)u_{x_{i}}+c(x)u
\end{equation}
or else 
\begin{equation}
    \label{eq:non_div-form_ellip}
    Lu=-\sum_{i,j=1}^{n}a^{ij}(x)u_{x_{i}x_{j}}+\sum_{i=1}^{n}b^{i}(x)u_{x_{i}}+c(x)u.
\end{equation}
The form \eqref{eq:div-form_ellip} is in \textbf{divergence form}, while the form \eqref{eq:non_div-form_ellip} is in \textbf{non-divergence form}. The requirement that $u=0$ on $\partial\Omega$ is called \textbf{Dirichlet's boundary condition}.

\begin{remark}
    \begin{itemize}
        \item Different with Evans chapter 2, $u\in C^{2}(U)$ is unnecessary. So we should discuss on the weak form of equation \eqref{eq:Elliptic_eq}.
        \item If $a^{ij}\in C^{2}(U)$, the form \eqref{eq:div-form_ellip} and \eqref{eq:non_div-form_ellip} are equivalent in general. But the form \eqref{eq:div-form_ellip} is more natural for energy methods, and the form \eqref{eq:non_div-form_ellip} is more appropriate for maximum principle techniques.
        \item Assumption: \textbf{symmetry condition}
        \begin{equation*}
            \label{eq:symmetry_cond}
            a^{ij}=a^{ji}.
        \end{equation*}
    \end{itemize}
\end{remark}
Now, give an important property of the differential operator $L$.
\begin{definition}
    \label{defn:uniformly_elliptic}
    We say the partial differential operator $L$ is (uniformly) elliptic if there exists a constant $\theta>0$ such that
    \begin{equation}
        \sum_{i,j=1}^{n}a^{ij}(x)\xi_{i}\xi_{j}\ge\theta |\xi|^{2}
    \end{equation}
    for a.e. $x\in U$ and all $\xi\in\mathbb{R}^{n}$.
\end{definition}

\begin{remark}
    \begin{itemize}
        \item $L$ is uniformly elliptic means the matrix $(a^{ij}(x))$ is positive definite $\forall x\in U$.
        \item The converse proposition of the above proposition isn't true.
        \item Special case: $a^{ij}(x)\equiv\delta_{ij}, b^{i}\equiv 0, c^{i}\equiv 0$. In this case, the equation \eqref{eq:Elliptic_eq} is the \textbf{Poisson equation}.
    \end{itemize}
\end{remark}
\subsection{Weak solutions}
\textbf{Motivation:}In general case, we can only assume that 
\begin{equation}
    \label{eq:assumption_for_ellip}
    a^{ij},b^{i},c\in L^{\infty}(U),f\in L^{2}(U).
\end{equation} 
In this case, maybe we can't find $u\in C^{2}(U)$ such that $u$ satisfies \eqref{eq:Elliptic_eq}, so we try to derive the \textbf{weak form} of the solution $u$, such that $u\in H_{0}^{1}(U)$. In practise, choose \textbf{test function} $v\in C_{c}^{\infty}(U)$, then define the linear functional 
\begin{equation}
    \label{eq:functional_of_u}
    u^{*}(v):=\int_{U}uv\dif x.
\end{equation}
Then consider the dual form of \eqref{eq:Elliptic_eq}, i.e. 
\begin{equation}
    \label{eq:dual_form}
    (Lu)^{*}(v)=f^{*}(v)\text{ }\forall v\in C_{c}^{\infty}(U).
\end{equation}
It's just the \textbf{weak form} of equation \eqref{eq:Elliptic_eq}. In this section, we choose the \textbf{divergence form} of operator $L$.
\begin{theorem}
    The weak form of equation \eqref{eq:Elliptic_eq} is 
    \begin{equation}
        \label{eq:weak_form}
        \int_{U}\sum_{i,j=1}^{n}a^{ij}u_{x_{i}}v_{x_{j}}+\sum_{i=1}^{n}b^{i}u_{x_{i}}v+cuv\dif x=\int_{U}fv\dif x.
    \end{equation}
\end{theorem}
\begin{proof}
    It's suffices to derive the expression of $\int_{U}vLu\dif x$. By Gauss-Green formula, for a vector field $F\in C^{1}(\mathbb{R}^{n})$, we can see:
    \begin{equation}
        \label{eq:Gauss-Green}
        \int_{U}v\nabla\cdot F\dif x=\int_{\partial U}v F\cdot n\dif S(x)-\int_{U}F\cdot\nabla v\dif x.
    \end{equation}
    Choose the vector field $F=\begin{bmatrix}
        \sum a^{i1}(x)u_{x_{i}}\\
        \sum a^{i2}(x)u_{x_{i}}\\
        \vdots\\
        \sum a^{in}(x)u_{x_{i}}\\
    \end{bmatrix}$, as $v=0$ on $\partial \Omega$, by \eqref{eq:Gauss-Green}, we can see:
    \begin{equation}
        \begin{aligned}
            \int_{U}\sum_{i,j=1}^{n}\left(a^{ij}(x)u_{x_{i}}\right)_{x_{j}}v\dif x&=\int_{U}v\nabla\cdot F\dif x\\
            &=-\int_{U}F\cdot\nabla v\dif x\\
            &=-\int_{U}\sum_{i,j=1}^{n}a^{ij}u_{x_{i}}v_{x_{j}}\dif x.\\
        \end{aligned}
    \end{equation}
    Then:
    \begin{equation}
        \label{eq:weak}
        \int_{U}vLu\dif x=\int_{U}\sum_{i,j=1}^{n}a^{ij}u_{x_{i}}v_{x_{j}}+\sum_{i=1}^{n}b^{i}u_{x_{i}}v+cuv\dif x=\int_{U}vf\dif x=f^{*}(v).
    \end{equation}
\end{proof}
By \eqref{eq:weak_form}, we can derive the following definitions.
\begin{definition}
    \begin{enumerate}
        \item The bilinear form $B[\;,\;]$ associated with the divergence form elliptic operator defined by \eqref{eq:div-form_ellip} is:
        \begin{equation}
            \label{eq:bilinear_form_B1}
            B[u,v]:=\int_{U}\sum_{i,j=1}^{n}a^{ij}u_{x_{i}}v_{x_{j}}+\sum_{i=1}^{n}b^{i}u_{x_{i}}v
        \end{equation}
        for $u,v\in H_{0}^{1}(U)$.
        \item We say that $u\in H_{0}^{1}(U)$ is a weak solution of the BVP \eqref{eq:Elliptic_eq} if 
        \begin{equation}
            B[u,v]=(f,v)
        \end{equation}
        for all $v\in H_{0}^{1}(U)$, where $(\;,\;)$ denotes the inner product in $L^{2}(U)$.
    \end{enumerate}
\end{definition}

More generally, let us consider the BVP 
\begin{equation}
    \label{eq:revised_BVP}
    \left\{
        \begin{aligned}
            Lu&=f^{0}-\sum_{i=1}^{n}f_{x_{i}}^{i},x\in U,\\
            u&=0,x\in\partial U,
        \end{aligned}
    \right.
\end{equation}
where $f^{i}\in L^{2}(U)$. Then we say $u\in H_{0}^{1}(U)$ is a weak solution of problem \eqref{eq:revised_BVP} if 
\begin{equation}
    B[u,v]=\innerprod{f}{v}
\end{equation}
for all $v\in H_{0}^{1}(U)$, where $\innerprod{\cdot}{\cdot}$ is the pairing of $H^{-1}(U)$ and $H_{0}^{1}(U)$.

For non-homogeneous elliptic PDE, i.e.
\begin{equation}
    \label{eq:nonhomo}
    \left\{
        \begin{aligned}
            Lu&=f,x\in U,\\
            u&=g,x\in \partial U.\\
        \end{aligned}
    \right.
\end{equation}
By trace theorem, $\exists w\in H^{1}(U)$ such that the trace of $w$ is $g$. Then define $\tilde{u}:=u-w$, \eqref{eq:nonhomo} is equivalent to the equation:
\begin{equation}
    \label{eq:change_to_homogeneous}
    \left\{
        \begin{aligned}
            L\tilde{u}&=\tilde{f},x\in U,\\
            \tilde{u}&=0,x\in\partial U,\\
        \end{aligned}
    \right.
\end{equation}
where $\tilde{f}:=f-Lw\in H^{-1}(U)$.
\section{Existence of weak solutions}
\subsection{Lax-Milgram theorem}
We now introduce an abstract principle from linear functional analysis. 

Assume $H$ is a real Hilbert space, with norm $\|\;\|$ and inner product $(\;,\;)$. We let $\innerprod{\cdot}{\cdot}$ denote the pairing of $H$ with its dual space.
\begin{theorem}{Lax-Milgram Theorem}
    \label{thm:Lax-Milgram}
    Assume that 
    \begin{equation}
        \label{eq:bilinear_form_B}
        B:H\times H\rightarrow\mathbb{R}
    \end{equation}
    is a bilinear mapping, for which there exist constants $\alpha,\beta$ such that 
    \begin{equation}
        \label{eq:bounded}
        |B(u,v)|\le \alpha \norm{u}\norm{v}(u,v\in H)
    \end{equation}
    and 
    \begin{equation}
        \label{eq:elliptic}
        \beta\norm{u}^{2}\le B[u,u](u\in H).
    \end{equation}
    Finally, let $f:H\rightarrow R$ be a bounded linear functional on $H$.

    Then there exists a unique element $u\in H$ such that 
    \begin{equation}
        \label{eq:Lax-Milgram}
        B[u,v]=\innerprod{f}{v}
    \end{equation}
    for all $v\in H$.
\end{theorem}
\begin{remark}
    \begin{enumerate}
        \item If $B$ is symmetry, the condition \eqref{eq:bounded} and \eqref{eq:elliptic} means $B$ can derive an inner product on $H$.
        \item So, if $B$ is symmetry, theorem \ref{thm:Lax-Milgram} is a direct corollary of Riesz representation theorem.
        \item If $B$ isn't symmetry, Riesz representation theorem can transform $f\in H^{*}$ to $u_{f}\in H$, such that $\innerprod{f}{v}=(u_{f},v)$.
        \item So, we should show that $\forall u_{f}\in H$, $\exists u\in H$ such that $B[u,v]=(u_{f},v)$ for each $v\in H$.
    \end{enumerate}
\end{remark}
\begin{proof}

    By the remark above, we just need to show that $\forall u_{f}\in H$, $\exists u\in H$ such that $B[u,v]=(u_{f},v)$ for each $v\in H$.

    Consider an element $u\in H$. $B(u,v)$ is a bounded bilinear mapping, so the map 
    \begin{equation}
        \label{eq:Bu}
        B_{u}(v):=B[u,v]
    \end{equation}
    is a bounded linear functional. By Riesz representation theorem, $\exists !\tilde{u}\in H$ such that $B[u,v]=(\tilde{u},v)$ $\forall v\in H$. So we can define a map $A:H\rightarrow H$ maps $u$ to $\tilde{u}$.
    
    Then it's suffices to show that $A$ is a bounded linear isomorphism.

    First we should show that $A$ is linear. $\forall v\in H$, $\lambda_{1},\lambda_{2}\in\mathbb{R}$, $u_{1},u_{2}\in H$, we can see:
    \begin{equation}
        \label{eq:linear_A}
        \begin{aligned}
        (A(\lambda_{1}u_{1}+\lambda_{2}u_{2}),v)&=B[\lambda_{1}u_{1}+\lambda_{2}u_{2},v]\\
        &=\lambda_{1}B[u_{1},v]+\lambda_{2}B[u_{2},v]\\
        &=\lambda_{1}(Au_{1},v)+\lambda_{2}(Au_{2},v).
        \end{aligned}
    \end{equation}
    \eqref{eq:linear_A} shows that $A$ is a linear map. 

    Then we show that $A$ is bounded. As $B$ is bounded, we can see:
    \begin{equation}
        \label{eq:Bounded_A}
        \begin{aligned}
        \norm{Au}^{2}&=(Au,Au)\\
        &=B[u,Au]\\
        &\le \alpha \norm{u}\norm{Au}.
        \end{aligned}
    \end{equation}
    i.e. $\norm{Au}\le\alpha\norm{u}$. So $A$ is bounded.

    The next thing to do in the proof is to show $A$ is an injective. By \eqref{eq:elliptic}, we can see:
    \begin{equation}
        \label{eq:injective_A}
        \begin{aligned}
        \beta\norm{u}^{2}&\le B[u,u]\\
        &=(Au,u)\\
        &\le \norm{Au}\norm{u}.
        \end{aligned}
    \end{equation}
    i.e. $\beta\norm{u}\le\norm{Au}$. So $\beta\norm{u}\le \norm{Au}\le\alpha\norm{u}$, it means that $A$ is an injective. What's more, the range of $A$, marked as $R(A)$, is a closed set.
    
    Finally, we should show that $A$ is a surjective. If $R(A)\neq H$, since $R(A)$ is closed, there exists a nonzero element  $\omega\in H$ with $\omega\in R(A)^{\bot}$. Then: 
    \begin{equation}
        \beta\norm{\omega}^{2}\le B[\omega,\omega]=(A\omega,\omega)=0.
    \end{equation} 
    which means that $\norm{\omega}=0$, contradict! So $A$ is a surjective.

    This completes the proof of Lax-Milgram theorem.
\end{proof}

Lax-Milgram theorem gives an important method for us to analyze the existence of weak solution.
\subsection{Energy estimates and First Existence theorem}
Now, we return to the specific bilinear form $B[\;,\;]$ defined by \eqref{eq:bilinear_form_B1}, and try to use \textbf{Lax-Milgram theorem} to prove the first existence theorem.
\begin{theorem}{Energy estimates}
\label{thm:energy_estimate}
There exists constants $\alpha,\beta>0$ and $\gamma\ge 0$ such that 
\begin{equation}
    \label{eq:bounded_B}
    |B[u,v]|\le\alpha\norm{u}_{H_{0}^{1}(U)}\norm{v}_{H_{0}^{1}(U)}
\end{equation}
and 
\begin{equation}
    \label{eq:elliptic_B}
    \beta\norm{u}_{H_{0}^{1}(U)}^{2}\le B[u,u]+\gamma\norm{u}_{L^{2}(U)}^{2}.
\end{equation}
\end{theorem}
\begin{proof}
    First derive the inequality \eqref{eq:bounded_B}. According to \eqref{eq:bilinear_form_B1}, we can check that 
    \begin{equation}
        \label{eq:derive_energy_B}
        \begin{aligned}
            B[u,v]&\le\sum_{i,j=1}^{n}\int_{U}\norm{a^{ij}}_{L^{\infty}}|u_{x_{i}}||v_{x_{j}}|\dif x+\sum_{i=1}^{n}\int_{U}\norm{b^{i}}_{L^{\infty}}|u_{x_{i}}||v|\dif x+\int_{U}\norm{c}_{L^{\infty}}|u||v|\dif x\\
            &\le\sum_{i,j=1}^{n}\int_{U}\norm{a^{ij}}_{L^{\infty}}|Du||Dv|\dif x+\sum_{i=1}^{n}\int_{U}\norm{b^{i}}_{L^{\infty}}|Du||v|\dif x+\int_{U}\norm{c}_{L^{\infty}}|u||v|\dif x\\
            &\le\alpha\norm{u}_{H_{0}^{1}(U)}\norm{v}_{H_{0}^{1}(U)}.
        \end{aligned}
    \end{equation}
    Then, by the uniformly elliptic condition of coefficient matrix $a^{ij}(x)$, we can see:
    \begin{equation}
        \label{eq:elliptic_for_a}
        \begin{aligned}
        \theta\int_{U}|Du|^{2}\dif x&\le \int_{U}\sum_{i,j=1}^{n}a^{ij}(x)u_{x_{i}}u_{x_{j}}\dif x\\
        &=B[u,u]-\int_{U}\left(\sum_{i=1}^{n}b^{i}(x)u_{x_{i}}u\dif x+cu^{2}\right)\dif x\\
        &\le B[u,u]+\sum_{i=1}^{n}\int_{U}\norm{b^{i}(x)}_{\Linf}|u||Du|\dif x+c\int_{U}u^{2}\dif x\\ 
        \end{aligned}
    \end{equation}
    By Cauchy-Schwarz inequality with coefficient, we can see:
    \begin{equation}
        \label{eq:cauchy_schwarz}
        \int_{U}|u||Du|\dif x\le\epsilon\int_{U}|Du|^{2}\dif x+\frac{1}{4\epsilon}\int_{U}|u|^{2}\dif x.
    \end{equation}
    Choose $\epsilon$ such that $\epsilon\sum_{i=1}^{n}\norm{b^{i}}_{L^{\infty}}<\frac{\theta}{2}$, then exists constant $C>0$ such that 
    \begin{equation}
        \label{eq:inequality_for_ellip}
        \frac{\theta}{2}\int_{U}|Du|^{2}\dif x\le B[u,u]+C\int_{U}u^{2}\dif x.
    \end{equation}
    Finally, by Poincare-Friedrichs inequality, the equation \eqref{eq:elliptic_B} is true.
\end{proof}
By \eqref{eq:elliptic_B}, if $\gamma>0$, $B[u,v]$ isn't uniformly elliptic in general. So, $B[u,v]$ does't satisfy the hypotheses of Lax-Milgram theorem in general. So we should give some revisions on bilinear form $B$. Then, we derive the \textbf{first existence theorem for weak solutions.}
\begin{theorem}{First Existence Theorem for weak solutions}
    \label{thm:1st_exist}
    There is a number $\gamma\ge 0$ such that for each $\mu\ge\gamma$ and each function $f\in L^{2}(U)$, there exists a unique weak solution $u\in H_{0}^{1}(U)$ of the boundary-value problem 
    \begin{equation}
        \label{eq:modified_BVP}
        \left\{
            \begin{aligned}
                Lu+\mu u&=f,x\in U;\\
                u&=0,x\in\partial\Omega.\\
            \end{aligned}
        \right.
    \end{equation}
\end{theorem}
\begin{proof}
    Choose the parameter $\gamma$ as theorem \ref{thm:energy_estimate}, then define the bilinear form 
    \begin{equation}
        \label{eq:Bmu}
        B_{\mu}[u,v]:=B[u,v]+\mu(u,v).
    \end{equation}
    Then the weak form of equation \eqref{eq:modified_BVP} is 
    \begin{equation}
        B_{\mu}[u,v]=\innerprod{f}{v}.
    \end{equation}
    By \eqref{eq:elliptic_B}, as $\mu\ge\gamma$, the bilinear form $B_{\mu}$ satisfies uniformly elliptic condition. So by Lax-Milgram theorem, equation \eqref{eq:Bmu} has unique weak solution.
\end{proof}
\begin{remark}
    The first existence theorem for weak solutions is a milestone, but we still can't give the existence theorem of equation \eqref{eq:Elliptic_eq}. We should use Fredholm alternative theorem to derive the existence of weak solution.
\end{remark}
\subsection{Fredholm alternative and the solvability}
In this section, we show the Fredholm alternative theorem first, then employ this theorem to derive the existence theorem for weak solutions.

\begin{theorem}{Fredholm alternative}
    \label{thm:Fredholm}
    Let $K:H\rightarrow H$ be a compact linear operator, then:
    \begin{itemize}
        \item $\ker(I-K)$ is finite dimensional.
        \item $R(I-K)$ is closed.
        \item $R(I-K)=\ker(I-K^{*})^{\perp}$.
        \item $\ker(I-K)=\{0\}$ if and only if $R(I-K)=H$.
        \item $\dim\ker(I-K)=\dim\ker(I-K^{*})$.
    \end{itemize}
\end{theorem}
\begin{proof}
    Omitted.
\end{proof}
Then, derive the \textbf{dual problem} of equation \eqref{eq:Elliptic_eq}.
\begin{definition}{Dual problem}
    \begin{enumerate}
        \item The operator $L^{*}$, the formal adjoint of $L$, is:
        \begin{equation}
            \label{eq:dual_op}
            L^{*}v=-\sum_{i,j=1}^{n}\left(a^{ij}v_{x_{j}}\right)_{x_{i}}-\sum_{i=1}^{n}b^{i}v_{x_{i}}+(c-\sum_{i=1}^{n}b_{x_{i}}^{i})v,
        \end{equation}
        provided $b^{i}\in C^{1}(\bar{U})$.
        \item The adjoint bilinear form 
        \begin{equation}
            \label{eq:dualB}
            B^{*}:H_{0}^{1}(U)\times H_{0}^{1}(U)\rightarrow\mathbb{R}
        \end{equation}
        is defined by 
        \begin{equation}
            \label{eq:dual_B}
            B^{*}[v,u]=B[u,v]
        \end{equation}
        for all $u,v\in H_{0}^{1}(U)$.
        \item We say that $v\in H_{0}^{1}(U)$ is a weak solution of the adjoint problem 
        \begin{equation}
            \label{eq:adj_problem}
            \left\{
                \begin{aligned}
                    L^{*}v&=f,x\in U;\\
                    v&=0,x\in\partial U.
                \end{aligned}
            \right.
        \end{equation}
        provided $B^{*}[v,u]=(f,u)$ for all $u\in H_{0}^{1}(U)$.
    \end{enumerate}
\end{definition}
\begin{remark}
    \begin{itemize}
        \item \eqref{eq:dual_op} is called formal adjoint, for $(Lu,v)=(u,L^{*}v)$.
        \item \eqref{eq:adj_problem} is the dual form of equation \eqref{eq:Elliptic_eq}.
    \end{itemize}
\end{remark}
To show the solvability of problem \eqref{eq:Elliptic_eq}, we derive the following theorem.
\begin{theorem}{Second Existence Theorem for weak solutions}
    \label{thm:2nd_exist}
    \begin{enumerate}
        \item Precisely one of the following statements holds: either for each $f\in L^{2}(U)$ there exists a unique weak solution $u$ of the boundary value problem \eqref{eq:Elliptic_eq} (marked as $\alpha$), or else there exists a weak solution $u\neq 0$ of the homogeneous problem (marked as $\beta$) 
        \begin{equation}
            \label{eq:homo_prob}
            \left\{
                \begin{aligned}
                    Lu&=0,x\in U;\\
                    u&=0,x\in \partial U.
                \end{aligned}
            \right.
        \end{equation}
        \item Furthermore, should assertion $(\beta)$ hold, the dimension of the subspace $N\subset H_{0}^{1}(U)$ of weak solutions of \eqref{eq:homo_prob} is finite and equals the dimension of the subspace $N^{*}\subset H_{0}^{1}(U)$ of weak solutions of 
        \begin{equation}
            \label{eq:homo_dual}
            \left\{
                \begin{aligned}
                    L^{*}v&=0,x\in U;\\
                    v&=0,x\in\partial U.\\
                \end{aligned}
            \right.
        \end{equation}
        \item Finally, the BVP \eqref{eq:Elliptic_eq} has a weak solution if and only if 
        \begin{equation}
            (f,v)=0\;\forall v\in N^{*}.
        \end{equation}
    \end{enumerate}
\end{theorem}
\begin{remark}
    To prove this theorem, we should try to use theorem \ref{thm:1st_exist}. The main idea is to construct a compact operator $K$, such that the weak form of \eqref{eq:Elliptic_eq} is $(I-K)u=h$, then use theorem \ref{thm:Fredholm}.
\end{remark}
\begin{proof}
    First, choose $\gamma$ as theorem \ref{thm:1st_exist} suggests, then for each $g\in L^{2}(U)$, there exists a unique $u\in H_{0}^{1}(U)$ solving:
    \begin{equation}
        \label{eq:auxiliary_weak_form}
        B_{\gamma}[u,v]=\innerprod{g}{v}\forall v\in H_{0}^{1}(U).
    \end{equation}
    Write $u=L_{\gamma}^{-1}g$ if equation \eqref{eq:auxiliary_weak_form} holds. As the weak form of \eqref{eq:Elliptic_eq} is $B[u,v]=\innerprod{f}{v}$, we can see $B_{\gamma}[u,v]=\innerprod{f+\gamma u}{v}$, i.e. $u=L_{\gamma}^{-1}(f+\gamma u)$. Then choose operator $K=\gamma L_{\gamma}^{-1}$, $h=L_{\gamma}^{-1}f$, \eqref{eq:auxiliary_weak_form} is equivalent to 
    \begin{equation}
        \label{eq:compact_weak_form}
        (I-K)u=h.
    \end{equation}

    The next step is to show that $K:L^{2}(U)\rightarrow L^{2}(U)$ is a bounded, linear, compact operator. In fact, we only need to show $K$ is compact. By \eqref{eq:auxiliary_weak_form}, we can see:
    \begin{equation}
        \label{eq:norm_estimate}
        \beta\norm{u}_{H_{0}^{1}(U)}^{2}\le B_{\gamma}[u,u]=\innerprod{g}{u}\le\norm{g}_{L^{2}(U)}\norm{u}_{L^{2}(U)}\le\norm{g}_{L^{2}(U)}\norm{u}_{H_{0}^{1}(U)}.
    \end{equation}
    By \eqref{eq:norm_estimate}, there exists a constant $C>0$ such that $\norm{Kg}_{H_{0}^{1}(U)}\le C\norm{g}_{L^{2}(U)}$. As $H_{0}^{1}(U)\subset\subset L^{2}(U)$, $K$ is a compact operator.

    Then, use theorem \ref{thm:Fredholm} on equation \eqref{eq:compact_weak_form}. If $\ker(I-K)=\{0\}$, theorem \ref{thm:Fredholm} shows that $I-K$ is also a surjective, i.e. statement $(\alpha)$ is true. Otherwise, $\ker(I-K)\neq\{0\}$ means that $\exists u\in H_{0}^{1}(U)$ such that $(I-K)u=0$, i.e. $u$ satisfies equation \eqref{eq:homo_prob}. Then statement $(\beta)$ is true. If $N\neq\{0\}$, while
    \begin{equation}\dim\ker(I-K)=\dim\ker(I-K^{*})<\infty,
    \end{equation} we can see $\dim N=\dim N^{*}<\infty$.

    Finally, if $(I-K)u=h$ has a solution, $v\in N^{*}$, we can see:
    \begin{equation}
        \innerprod{h}{v}=\innerprod{(I-K)u}{v}=\innerprod{u}{(I-K^{*})v}=0.
    \end{equation}
    By $h=L_{\gamma}^{-1}f$, we can see:
    \begin{equation}
        \innerprod{h}{v}=\frac{1}{\gamma}\innerprod{Kf}{v}=\frac{1}{\gamma}\innerprod{f}{K^{*}v}=\frac{1}{\gamma}\innerprod{f}{v}=0.
    \end{equation}
    So:$\innerprod{f}{v}=0$.
\end{proof}
\subsection{Spectrum and third existence theorem}
In this section, we will discuss on the \textbf{spectrum} of an operator $L$, then derive the existence of weak solutions for eigenvalue problem.
\begin{theorem}{Third existence theorem for weak solutions}
    \begin{enumerate}
        \item There exists an at most countable set $\Sigma\subset\mathbb{R}$ such that the BVP 
        \begin{equation}
            \label{eq:eigen-BVP}
            \left\{
                \begin{aligned}
                    Lu&=\lambda u+f,x\in U;\\
                    u&=0,x\in\partial U\\
                \end{aligned}
            \right.
        \end{equation}
        has a unique weak solution for each $f\in L^{2}(U)$ if and only if $\lambda\notin\Sigma$.
        \item If $\Sigma$ is infinite, then $\Sigma=\{\lambda_{k}\}_{k=1}^{\infty}$, the values of a nondecreasing sequence with $\lambda_{k}\rightarrow+\infty$.
    \end{enumerate}
\end{theorem}
\begin{remark}
    \begin{enumerate}
        \item $\Sigma$ is called the \textbf{spectrum} of operator $L$.
        \item If $f=0$, the BVP \eqref{eq:eigen-BVP} is called eigenvalue problem, and if there exists a solution $\omega\neq 0$, $\lambda$ is called an \textbf{eigenvalue} of $L$, and $\omega$ is a corresponding \textbf{eigenfunction}. 
    \end{enumerate}
\end{remark}
\begin{proof}
    By theorem \ref{thm:2nd_exist}, if $\lambda\in\Sigma$, the homogeneous eigenvalue problem 
    \begin{equation}
        \label{eq:homo-eig-prob}
        \left\{
            \begin{aligned}
            Lu&=\lambda u,x\in U,\\
            u&=0,x\in\partial U
            \end{aligned}
        \right.
    \end{equation}
    has a solution $u\neq 0$. Consider it's weak form, we can see:
    \begin{equation}
        \label{eq:weak_form_for_eig_BVP}
        B_{\gamma}[u,v]=(\lambda+\gamma)\innerprod{u}{v}.
    \end{equation}
    i.e.
    \begin{equation}
        u=L_{\gamma}^{-1}(\gamma+\lambda)u=\frac{\lambda+\gamma}{\gamma}Ku.
    \end{equation}
    As $u\neq 0$, $u$ is the eigenvector of operator $K$, the corresponding eigenvalue is $\frac{\gamma}{\lambda+\gamma}$. By theorem \ref{thm:2nd_exist}, $K$ is a compact operator, so the Spectrum set $S$ of operator $K$ is either finite set, or else the values of a sequence converging to zero. It means that $\Sigma$ is at most countable, and if $|\Sigma|=\infty$, $\lambda_{k}\rightarrow\infty$.
\end{proof}

Finally, we note the boundedness of eigenvalue problem.
\begin{theorem}{Boundedness of the inverse}
If $\lambda\notin\Sigma$, there exists a constant $C$ such that 
\begin{equation}
    \label{eq:bounded}
    \norm{u}_{L^{2}(U)}\le C\norm{f}_{L^{2}(U)}
\end{equation}
whenever $f\in L^{2}(U)$ and $u$ is the unique weak solution of \eqref{eq:eigen-BVP}. The constant $C$ depends only on $\lambda,U$ and the coefficients of $L$.
\end{theorem}
\begin{proof}
    If not, there would exist sequences $\{f_{k}\}\subset L^{2}(U)$ and $\{u_{k}\}\subset H_{0}^{1}(U)$ such that 
    \begin{equation}
        \label{eq:counter}
        \left\{
            \begin{aligned}
        Lu_{k}&=\lambda u_{k}+f_{k},x\in U,\\
        u_{k}&=0,x\in \partial U,\\
            \end{aligned}
        \right.
    \end{equation}
    but 
    \begin{equation}
        \norm{u_{k}}_{L^{2}(U)}>k\norm{f_{k}}_{L^2(U)}.
    \end{equation}
    Assume WLOG, $\norm{u_{k}}_{L^2(U)}=1$, we can see $\norm{f_{k}}\rightarrow 0$. Then there exists a subsequence $\{u_{k_{j}}\}$ satisfies:
    \begin{equation}
        \label{eq:convergence}
        \left\{
            \begin{aligned}
                u_{k_{j}}&\rightharpoonup u\text{ in }H_{0}^{1}(U),\\
                u_{k_{j}}&\rightarrow u\text{ in }L^{2}(U).\\
            \end{aligned}
        \right.
    \end{equation}
    Then $u$ is a weak solution of \eqref{eq:homo-eig-prob}. Since $\lambda\notin\Sigma$, $u\equiv 0$. But $\norm{u_{k}}_{L^{2}(U)}\equiv 1$, contradict!
\end{proof}
\section{Interior Regularity}
Now, we try to discuss on the \textbf{regularity} of weak solutions. Consider a general PDE $Lu=f$, we find the weak solution $u\in H^{1}(U)$. However, if we set $f\in H^{m}(U)$, we expect $u\in H^{m+2}(U)$, it means we derive stronger regularity about the weak solution of $Lu=f$. The point of regularity is to derive analytic estimates from the structural, algebraic assumption of ellipticity.

First, recall the definition of difference quotients, and some related properties.
\begin{definition}{Difference quotient}
    \label{def:diff_quo}
    The $i$-th difference quotient of size $h$ is 
    \begin{equation}
        \label{eq:diff_quo}
        D_{i}^{h}u(x)=\frac{u(x+he_{i})-u(x)}{h}(i=1,\cdots,n)
    \end{equation}
    for $x\in V$ and $h\in\mathbb{R}$, $0<|h|<\text{dist}(V,\partial U)$. And $D^{h}(u):=\left(D_{1}^{h}u,\cdots,D_{n}^{h}u\right)$.
\end{definition}
The concept of difference quotients is related to weak derivatives, as the following theorem:
\begin{theorem}
    \label{thm:diff_and_dir}
    \begin{enumerate}
        \item Suppose $1\le p<\infty$ and $u\in W^{1,p}(U)$, for each $V\subset\subset U$, 
        \begin{equation}
            \norm{D^{h}u}_{L^{p}(V)}\le C\norm{Du}_{L^{p}(U)}
        \end{equation}
        for some constant $C$ and all $0<|h|<\frac{1}{2}\text{dist}(V,\partial U).$
        \item Assume $1<p<\infty$, $u\in L^{p}(V)$, and there exists a constant $C$ such that 
        \begin{equation}
            \norm{D^{h}u}_{L^{p}(V)}\le C
        \end{equation}
        for all $0<|h|<\frac{1}{2}\text{dist}(V,\partial U)$, then $u\in W^{1,p}(V)$ with $\norm{Du}_{L^{p}(V)}\le C$.        
    \end{enumerate}
\end{theorem}
\begin{proof}
    see \cite{evans2022partial} section 5.8, theorem 3.
\end{proof}

Then, introduce two lemmas for the integrate of difference quotients.
\begin{lemma}
    \label{lem:diff_quo}
    For a bounded open set $U$, and open set $W\subset\subset U$, 
    assume $v,w\in H^{1}(U)$,  $\text{supp}(w)\subset W$, $h<\frac{1}{2}\text{dist}(W,\partial U)$, then we can see:
    \begin{equation}
        \label{eq:int_by_part}
        \int_{U}vD_{k}^{-h}w\dif x=-\int_{U}wD_{k}^{h}v\dif x,
    \end{equation}
    and  
    \begin{equation}
        \label{eq:leibniz_for_diff}
        D_{k}^{h}(vw)=v^{h}D_{k}^{h}w+wD_{k}^{h}v,
    \end{equation}
    for $v^{h}(x):=v(x+he_{k})$.
\end{lemma}
\begin{proof}
    For equation \eqref{eq:int_by_part}, as $\text{supp} w\subset W$, we can see:
    \begin{equation}
        \begin{aligned}
        RHS&=-\int_{W}w(x)\frac{v(x+he_{k})-v(x)}{h}\dif x\\
        &=\frac{1}{h}\int_{W}w(x)v(x)\dif x-\frac{1}{h}\int_{W}w(x)v(x+he_{k})\dif x.
        \end{aligned}
    \end{equation}
    set $\tilde{W}:=\{y:y=x+he_{k},x\in W\}$, we can see:
    \begin{equation}
        \begin{aligned}
            LHS&=\frac{1}{h}\int_{U}v(x)\left(w(x)-w(x+he_{k})\right)\dif x\\
            &=\frac{1}{h}\int_{W}v(x)w(x)\dif x-\frac{1}{h}\int_{\tilde{W}}v(x)w(x-he_{k})\dif x.
        \end{aligned}
    \end{equation}
    By integral substitution, choose $y:=x-he_{k}$, we can see:
    \begin{equation}
        \int_{\tilde{W}}v(x)w(x-he_{k})\dif x=\int_{W}v(y+he_{k})w(y)\dif y.
    \end{equation}
    So, equation \eqref{eq:int_by_part} is true. Then, for equation \eqref{eq:leibniz_for_diff}, we can see:
    \begin{equation}
        \begin{aligned}
            LHS&=\frac{v(x+he_{k})w(x+he_{k})-v(x)w(x)}{h}\\
            &=\frac{v(x+he_{k})(w(x+he_{k})-w(x))+w(x)(v(x+he_{k})-v(x))}{h}\\
            &=v^{h}D_{k}^{h}w+wD_{k}^{h}v=RHS.
        \end{aligned}
    \end{equation}
\end{proof}
Then, we give the main result for this section:
\begin{theorem}{Interior $H^{2}$-regularity}
    \label{thm:interior_regularity}
    Assume $a^{ij}\in C^{1}(U)$, $b^{i},c\in L^{\infty}(U)$, and $f\in L^{2}(U)$. Suppose furthermore that $u\in H^{1}(U)$ is a weak solution of the elliptic PDE $Lu=f$, then:
    \begin{equation}
        u\in H_{loc}^{2}(U);
    \end{equation}
    and for each open subset $V\subset\subset U$ we have the estimate 
    \begin{equation}
        \label{eq:estimate_H2_u}
        \norm{u}_{H^{2}(V)}\le C\left(\norm{f}_{L^{2}(U)}+\norm{u}_{L^{2}(U)}\right),
    \end{equation}
    the constant $C$ depending only on $V$, $U$, and the coefficients of operator $L
    $.
\end{theorem}
Before the proof of this theorem, we should give some remarks.

\begin{remark}
    \begin{enumerate}
        \item In this theorem, we don't require $u\in H_{0}^{1}(U)$.
        \item Since $u\in H_{loc}^{2}(U)$, we have
        \begin{equation}
            Lu=f\text{ a.e. in U}.
        \end{equation}
    \end{enumerate}
\end{remark}

\textbf{The idea} : First, construct a truncated function $\zeta\in C^{\infty}(U)$ such that $V\subset\subset W\subset\subset U$, $\zeta|_{V}\equiv 1$, $\zeta|_{U\setminus W}\equiv 0$. Then, choose a test function 
\begin{equation}
    \label{eq:test_func}
    v(x)=-D_{k}^{-h}\left(\zeta^{2}D_{k}^{h}u(x)\right)\in H_{0}^{1}(U).
\end{equation}
In fact, if $u\in C^{2}(U)$, $v$ is a difference quotient approximation for $D^{2}u$. Finally, use the relation $B[u,v]=(f,v)$ to approximate $\norm{D_{k}^{h}Du}_{L^{2}(V)}$, and use the second part of theorem \ref{thm:diff_and_dir}.
\begin{proof}
    By the definition of weak solution, $B[u,v]=(f,v)$. Then we can see:
    \begin{equation}
        \label{eq:modified_weak_sol}
        \sum \int_{U}a^{ij}u_{x_{i}}v_{x_{j}}\dif x=\int_{U}(f-\sum b_{i}u_{x_{i}}-cu)v\dif x.
    \end{equation}
    mark 
    \begin{equation}
        \label{eq:expression_for_A}
        A=\sum\int_{U}a^{ij}u_{x_{i}}v_{x_{j}}\dif x,
    \end{equation}
    and choose $v$ as \eqref{eq:test_func}, according to lemma \ref{lem:diff_quo}, we can derive:
    \begin{equation}
        \label{eq:partition_for_A}
        \begin{aligned}
        A&=-\int_{U}\sum a^{ij}u_{x_{i}}\left(D_{k}^{-h}\left(\zeta^{2}D_{k}^{h}u(x)\right)\right)_{x_{j}}\dif x\\
        &=\sum\int_{U}D_{k}^{h}\left(a^{ij}u_{x_{i}}\right)\left(\zeta^{2}D_{k}^{h}u(x)\right)_{x_{j}}\dif x\\
        &=\sum\int_{U}\left(a^{ij,h}D_{k}^{h}u_{x_{i}}+u_{x_{i}}D_{k}^{h}(a^{ij})\right)\left(2\zeta\zeta_{x_{j}}D_{k}^{h}u(x)+\zeta^{2}\left(D_{k}^{h}u(x)\right)_{x_{j}}\right)\dif x\\
        &=\sum\int_{U}a^{ij,h}\zeta^{2}D_{k}^{h}u_{x_{i}}D_{k}^{h}u_{x_{j}}\dif x\\
        &+\sum\int_{U}\left(2\zeta\zeta_{x_{j}}u_{x_{i}}D_{k}^{h}(a^{ij})D_{k}^{h}u(x)+2\zeta\zeta_{x_{j}}a^{ij,h}D_{k}^{h}u_{x_{i}}D_{k}^{h}u(x)+\zeta^{2}u_{x_{i}}D_{k}^{h}(a^{ij})\left(D_{k}^{h}u(x)\right)_{x_{j}}\right)\dif x\\
        &:=A_{1}+A_{2}.
        \end{aligned}
    \end{equation}
    By the uniform elliptic property, there exists $\theta>0$ such that 
    \begin{equation}
        \label{eq:approx_A1}
        A_{1}\ge \theta\int_{U}\zeta^{2}|D_{k}^{h}Du|^{2}\dif x.
    \end{equation}
    Then we try to approx $|A_{2}|$. As $a^{ij}\in C^{1}(U)$, $\zeta\in C^{\infty}(U)$, we can see:
    \begin{equation}
        \label{eq:approx_A2}
        \begin{aligned}
            |A_{2}|&\le \sum\int_{U}\left(C_{1}\zeta|u_{x_{i}}||u(x)|+C_{2}\zeta|D_{k}^{h}u_{x_{i}}||D_{k}^{h}u(x)|+C_{3}\zeta|u_{x_{i}}|\left(D_{k}^{h}u(x)\right)_{x_{j}}\right)\dif x\\
            &\le C\int_{U}\left(\zeta|Du||u|+\zeta|D_{k}^{h}Du||D_{k}^{h}u|+\zeta|Du||D_{k}^{h}Du|\right)\dif x\\
            &\le \frac{\theta}{2} \int_{U}\zeta^{2}|D_{k}^{h}Du|^{2}\dif x+\tilde{C}\int_{U}|Du|^{2}+|D_{k}^{h}u|^{2}\dif x\\
            &\le\frac{\theta}{2}\int_{U}\zeta^{2}|D_{k}^{h}Du|^{2}\dif x+M\int_{U}|Du|^{2}\dif x.
        \end{aligned}
    \end{equation}
    In equation \eqref{eq:approx_A2}, $C_{1},C_{2},C_{3},C,\tilde{C},M$ are all constants. The first step follows from $a^{ij},\zeta\in C^{1}(U)$, the second and the third steps from Cauchy-Schwarz inequality, and the last from theorem \ref{thm:diff_and_dir}.

    Combining \eqref{eq:approx_A1} and \eqref{eq:approx_A2}, we can see 
    \begin{equation}
        \label{eq:approx_A}
        A\ge \frac{\theta}{2}\int_{U}\zeta^{2}|D_{k}^{h}Du|^{2}\dif x-M\int_{U}|Du|^{2}\dif x.
    \end{equation}

    The next step is to approximate the right-hand integral 
    \begin{equation}
        \label{eq:integral_B}
        B:=\int_{U}(f-\sum b_{i}u_{x_{i}}-cu)v\dif x.
    \end{equation}
    First, as $b_{i}\in C^{1}(U)$, we can see 
    \begin{equation}
        \label{eq:approx1_B}
        |B|\le C\int_{U}(|f|+|Du|+|u|)|v|\dif x.
    \end{equation}
    Then, consider 
    \begin{equation}
        \label{eq:approx_int_v2}
        \begin{aligned}
            \int_{U}v^{2}\dif x&=\int_{U}|D_{k}^{-h}(\zeta^{2}D_{k}^{h}u)|^{2}\dif x\\
            &\le C\int_{U}D|\zeta^{2}D_{k}^{h}u|^{2}\dif x\\
            &\le C\int_{U}\zeta^{2}|D_{k}^{h}Du|^{2}\dif x+C\int_{W}|D_{k}^{h}u|^{2}\dif x\\
            &\le C\int_{U}\left(|Du|^{2}+\zeta^{2}|D_{k}^{h}Du|^{2}\right)\dif x.
        \end{aligned}
    \end{equation}
    The second step follows from theorem \ref{thm:diff_and_dir}, the third step follows from the Leibniz formula, and the final step follows from theorem \ref{thm:diff_and_dir} as well.
    
    Finally, by \eqref{eq:approx1_B} , \eqref{eq:approx_int_v2} and Cauchy-Schwarz inequality, there exists constant $C$ such that 
    \begin{equation}
        \label{eq:approx_B}
        |B|\le\frac{\theta}{4}\int_{U}\zeta^{2}|D_{k}^{h}Du|^{2}\dif x+C\int_{U}\left(|f|^2+|u|^2+|Du|^2\right)\dif x.
    \end{equation}
    
    By the estimation \eqref{eq:approx_A} and \eqref{eq:approx_B}, it's clear that $\forall h>0$, $\exists$ constant $C>0$, such that 
    \begin{equation}
        \int_{V}|D_{k}^{h}Du|^{2}\dif x\le C(\norm{f}_{L^{2}(U)}+\norm{u}_{H^{1}(U)}).
    \end{equation}
    It means that $Du\in H^{1}(V)$, and 
    \begin{equation}
        \norm{u}_{H^{2}(V)}\le C\left(\norm{f}_{L^{2}(U)}+\norm{u}_{H^{1}(U)}\right).
    \end{equation}

    Finally, choose auxiliary function $\xi\in C^{\infty}(U)$, $\text{supp}\xi\subset U$ and $\xi\equiv 1$ on $W$, set $v=\xi^{2}u$, according to equation \eqref{eq:modified_weak_sol}, we can see:
    \begin{equation}
        \int_{U}\xi^{2}|Du|^{2}\dif x\le C\int_{U}(f^{2}+u^2)\dif x.
    \end{equation}
    Then:
    \begin{equation}
        \norm{u}_{H^{1}(W)}\le C\left(\norm{f}_{L^{2}(U)}+\norm{u}_{L^{2}(U)}\right).
    \end{equation}
\end{proof}
\begin{remark}
    Theorem \ref{thm:interior_regularity} shows the condition for the higher regularity of weak solutions.
\end{remark}
Then, we introduce the higher interior regularity.
\begin{theorem}{Higher interior regularity}
    \label{thm:higher_int_regular}
    Let $m$ be a nonnegative integer, and assume $a^{ij},b^{i},c\in C^{m+1}(U)$, and $f\in H^{m}(U)$. Suppose $u\in H^{1}(U)$ is a weak solution of the elliptic PDE
    \begin{equation}
        Lu=f,
    \end{equation}
    then $u\in H_{loc}^{m+2}(U)$ and for each $V\subset\subset U$ we have the estimate 
    \begin{equation}
        \norm{u}_{H^{m+2}(V)}\le C(\norm{f}_{H^{m}(U)}+\norm{u}_{L^{2}(U)}).
    \end{equation}
\end{theorem}
\begin{proof}
    We prove this theorem by induction. If $m=0$, theorem \ref{thm:higher_int_regular} is equivalent to theorem \ref{thm:interior_regularity}, so the induction basis is true. Assume theorem \ref{thm:higher_int_regular} is true for $m=k$, i.e. if $a^{ij},b^{i},c\in C^{k+1}(U)$, $f\in H^{k}(U)$, the weak solution $u\in H^{1}(U)$ satisfies
    \begin{equation}
        \norm{u}_{H^{k+2}(W)}\le C(\norm{f}_{H^{k}(U)}+\norm{u}_{L^{2}(U)}).
    \end{equation}
    for each $W\subset\subset U$ and approximate constant $C$, and $u\in H_{loc}^{k+2}(U)$. Then let $\alpha$ be any multiindex with $|\alpha|=m+1$ and choose any auxiliary function $\tilde{v}\in C_{c}^{\infty}(W)$. Choose the test function $v=(-1)^{|\alpha|}D^{\alpha}\tilde{v}$, insert it into the weak form $B[u,v]=(f,v)$, perform some integrations by parts, we discover:
    \begin{equation}
        \label{eq:Buv_int_by_part}
        \begin{aligned}
            B[u,v]&=\sum\int_{U}a^{ij}u_{x_{i}}(-1)^{|\alpha|}(D^{\alpha}\tilde{v})_{x_{j}}\dif x+\sum\int_{U}b^{i}u_{x_{i}}(-1)^{|\alpha|}D^{\alpha}\tilde{v}\dif x+\int_{U}cu(-1)^{|\alpha|}D^{\alpha}\tilde{v}\dif x\\
            &=\sum\int_{U}D^{\alpha}(a^{ij}u_{x_{i}})\tilde{v}_{x_{j}}\dif x+\sum\int_{U}D^{\alpha}(b^{i}u_{x_{i}})\tilde{v}\dif x+\int_{U}D^{\alpha}(cu)\tilde{v}\dif x\\
            &=B[D^{\alpha}u,\tilde{v}]-\sum_{i,j}\sum_{\beta\le\alpha}\int_{U}\binom{\alpha}{\beta}(D^{\alpha-\beta}a^{ij}D^{\beta}u_{x_{i}})_{x_{j}}\tilde{v}\dif x+\sum_{i}\sum_{\beta\le\alpha}\int_{U}D^{\alpha-\beta}b^{i}D^{\beta}u_{x_{i}}\tilde{v}\dif x\\
            &+\sum_{\beta\le\alpha}\int_{U}D^{\alpha-\beta}cD^{\beta}u\tilde{v}\dif x.
        \end{aligned}
    \end{equation}
    So if we write $\tilde{u}:=D^{\alpha}u$, we can see:
    \begin{equation}
        B[\tilde{u},\tilde{v}]=(\tilde{f},\tilde{v}).
    \end{equation}
    Where 
    \begin{equation}
        \tilde{f}:=D^{\alpha}f-\sum_{\beta<\alpha}\binom{\alpha}{\beta}\left[-\sum(D^{\alpha-\beta}a^{ij}D^{\beta}u_{x_{i}})_{x_{j}}+\sum D^{\alpha-\beta}b^{i}D^{\beta}u_{x_{i}}+D^{\alpha-\beta}cD^{\beta}u\right].
    \end{equation}
    We can see that $\tilde{u}$ is the weak solution of equation $L\tilde{u}=\tilde{f}$. Since $a^{ij},b^{i},c\in C^{k+2}(U)$, $f\in H^{k+1}(U)$, we can see $\tilde{f}\in L^{2}(W)$, and 
    \begin{equation}
        \norm{\tilde{f}}_{L^{2}(W)}\le C\left(\norm{f}_{H^{m+1}(U)}+\norm{u}_{L^{2}(U)}\right).
    \end{equation}
    In light of theorem \ref{thm:interior_regularity}, we see $\tilde{u}\in H^{2}(V)$ with 
    \begin{equation}
        \norm{\tilde{u}}_{H^{2}(V)}\le\ C(\norm{f}_{H^{m+1}(U)}+\norm{u}_{L^{2}(U)}).
    \end{equation}
    As the inequality holds for all multiindex $\alpha$ satisfies $|\alpha|=m+1$, the theorem is true.
\end{proof}