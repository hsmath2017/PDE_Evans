\chapter{Elliptic equation}
In this chapter, we will discuss on the general elliptic PDEs and its weak form. We will exploit two essentially distinct techniques, energy methods within Sobolev spaces and maximum principle methods.
\section{Definition}
\subsection{Elliptic equations}
We will in this chapter mostly study the boundary value problem(BVP):
\begin{equation}
    \label{eq:Elliptic_eq}
    \left\{
        \begin{aligned}
            Lu&=f\text{ in }U;\\
            u&=0\text{ on }\partial U,\\
        \end{aligned}
    \right.
\end{equation} 
where $U$ is an open, bounded subset of $\mathbb{R}^{n}$, and $u:\bar{U}\rightarrow\mathbb{R}$ is the unknown. Here $f$ is given, and $L$ denotes a second order differential operator have either the form:
\begin{equation}
    \label{eq:div-form_ellip}
    Lu=-\sum_{i,j=1}^{n}\left(a^{ij}(x)u_{x_{i}}\right)_{x_{j}}+\sum_{i=1}^{n}b^{i}(x)u_{x_{i}}+c(x)u
\end{equation}
or else 
\begin{equation}
    \label{eq:non_div-form_ellip}
    Lu=-\sum_{i,j=1}^{n}a^{ij}(x)u_{x_{i}x_{j}}+\sum_{i=1}^{n}b^{i}(x)u_{x_{i}}+c(x)u.
\end{equation}
The form \eqref{eq:div-form_ellip} is in \textbf{divergence form}, while the form \eqref{eq:non_div-form_ellip} is in \textbf{non-divergence form}. The requirement that $u=0$ on $\partial\Omega$ is called \textbf{Dirichlet's boundary condition}.

\begin{remark}
    \begin{itemize}
        \item Different with Evans chapter 2, $u\in C^{2}(U)$ is unnecessary. So we should discuss on the weak form of equation \eqref{eq:Elliptic_eq}.
        \item If $a^{ij}\in C^{2}(U)$, the form \eqref{eq:div-form_ellip} and \eqref{eq:non_div-form_ellip} are equivalent in general. But the form \eqref{eq:div-form_ellip} is more natural for energy methods, and the form \eqref{eq:non_div-form_ellip} is more appropriate for maximum principle techniques.
        \item Assumption: \textbf{symmetry condition}
        \begin{equation*}
            \label{eq:symmetry_cond}
            a^{ij}=a^{ji}.
        \end{equation*}
    \end{itemize}
\end{remark}
Now, give an important property of the differential operator $L$.
\begin{definition}
    \label{defn:uniformly_elliptic}
    We say the partial differential operator $L$ is (uniformly) elliptic if there exists a constant $\theta>0$ such that
    \begin{equation}
        \sum_{i,j=1}^{n}a^{ij}(x)\xi_{i}\xi_{j}\ge\theta |\xi|^{2}
    \end{equation}
    for a.e. $x\in U$ and all $\xi\in\mathbb{R}^{n}$.
\end{definition}

\begin{remark}
    \begin{itemize}
        \item $L$ is uniformly elliptic means the matrix $(a^{ij}(x))$ is positive definite $\forall x\in U$.
        \item The converse proposition of the above proposition isn't true.
        \item Special case: $a^{ij}(x)\equiv\delta_{ij}, b^{i}\equiv 0, c^{i}\equiv 0$. In this case, the equation \eqref{eq:Elliptic_eq} is the \textbf{Poisson equation}.
    \end{itemize}
\end{remark}
\subsection{Weak solutions}
\textbf{Motivation:}In general case, we can only assume that 
\begin{equation}
    \label{eq:assumption_for_ellip}
    a^{ij},b^{i},c\in L^{\infty}(U),f\in L^{2}(U).
\end{equation} 
In this case, maybe we can't find $u\in C^{2}(U)$ such that $u$ satisfies \eqref{eq:Elliptic_eq}, so we try to derive the \textbf{weak form} of the solution $u$, such that $u\in H_{0}^{1}(U)$. In practise, choose \textbf{test function} $v\in C_{c}^{\infty}(U)$, then define the linear functional 
\begin{equation}
    \label{eq:functional_of_u}
    u^{*}(v):=\int_{U}uv\dif x.
\end{equation}
Then consider the dual form of \eqref{eq:Elliptic_eq}, i.e. 
\begin{equation}
    \label{eq:dual_form}
    (Lu)^{*}(v)=f^{*}(v)\text{ }\forall v\in C_{c}^{\infty}(U).
\end{equation}
It's just the \textbf{weak form} of equation \eqref{eq:Elliptic_eq}. In this section, we choose the \textbf{divergence form} of operator $L$.
\begin{theorem}
    The weak form of equation \eqref{eq:Elliptic_eq} is 
    \begin{equation}
        \label{eq:weak_form}
        \int_{U}\sum_{i,j=1}^{n}a^{ij}u_{x_{i}}v_{x_{j}}+\sum_{i=1}^{n}b^{i}u_{x_{i}}v+cuv\dif x=\int_{U}fv\dif x.
    \end{equation}
\end{theorem}
\begin{proof}
    It's suffices to derive the expression of $\int_{U}vLu\dif x$. By Gauss-Green formula, for a vector field $F\in C^{1}(\mathbb{R}^{n})$, we can see:
    \begin{equation}
        \label{eq:Gauss-Green}
        \int_{U}v\nabla\cdot F\dif x=\int_{\partial U}v F\cdot n\dif S(x)-\int_{U}F\cdot\nabla v\dif x.
    \end{equation}
    Choose the vector field $F=\begin{bmatrix}
        \sum a^{i1}(x)u_{x_{i}}\\
        \sum a^{i2}(x)u_{x_{i}}\\
        \vdots\\
        \sum a^{in}(x)u_{x_{i}}\\
    \end{bmatrix}$, as $v=0$ on $\partial \Omega$, by \eqref{eq:Gauss-Green}, we can see:
    \begin{equation}
        \begin{aligned}
            \int_{U}\sum_{i,j=1}^{n}\left(a^{ij}(x)u_{x_{i}}\right)_{x_{j}}v\dif x&=\int_{U}v\nabla\cdot F\dif x\\
            &=-\int_{U}F\cdot\nabla v\dif x\\
            &=-\int_{U}\sum_{i,j=1}^{n}a^{ij}u_{x_{i}}v_{x_{j}}\dif x.\\
        \end{aligned}
    \end{equation}
    Then:
    \begin{equation}
        \label{eq:weak}
        \int_{U}vLu\dif x=\int_{U}\sum_{i,j=1}^{n}a^{ij}u_{x_{i}}v_{x_{j}}+\sum_{i=1}^{n}b^{i}u_{x_{i}}v+cuv\dif x=\int_{U}vf\dif x=f^{*}(v).
    \end{equation}
\end{proof}
By \eqref{eq:weak_form}, we can derive the following definitions.
\begin{definition}
    \begin{enumerate}
        \item The bilinear form $B[\;,\;]$ associated with the divergence form elliptic operator defined by \eqref{eq:div-form_ellip} is:
        \begin{equation}
            \label{eq:bilinear_form_B1}
            B[u,v]:=\int_{U}\sum_{i,j=1}^{n}a^{ij}u_{x_{i}}v_{x_{j}}+\sum_{i=1}^{n}b^{i}u_{x_{i}}v
        \end{equation}
        for $u,v\in H_{0}^{1}(U)$.
        \item We say that $u\in H_{0}^{1}(U)$ is a weak solution of the BVP \eqref{eq:Elliptic_eq} if 
        \begin{equation}
            B[u,v]=(f,v)
        \end{equation}
        for all $v\in H_{0}^{1}(U)$, where $(\;,\;)$ denotes the inner product in $L^{2}(U)$.
    \end{enumerate}
\end{definition}

More generally, let us consider the BVP 
\begin{equation}
    \label{eq:revised_BVP}
    \left\{
        \begin{aligned}
            Lu&=f^{0}-\sum_{i=1}^{n}f_{x_{i}}^{i},x\in U,\\
            u&=0,x\in\partial U,
        \end{aligned}
    \right.
\end{equation}
where $f^{i}\in L^{2}(U)$. Then we say $u\in H_{0}^{1}(U)$ is a weak solution of problem \eqref{eq:revised_BVP} if 
\begin{equation}
    B[u,v]=\innerprod{f}{v}
\end{equation}
for all $v\in H_{0}^{1}(U)$, where $\innerprod{\cdot}{\cdot}$ is the pairing of $H^{-1}(U)$ and $H_{0}^{1}(U)$.

For non-homogeneous elliptic PDE, i.e.
\begin{equation}
    \label{eq:nonhomo}
    \left\{
        \begin{aligned}
            Lu&=f,x\in U,\\
            u&=g,x\in \partial U.\\
        \end{aligned}
    \right.
\end{equation}
By trace theorem, $\exists w\in H^{1}(U)$ such that the trace of $w$ is $g$. Then define $\tilde{u}:=u-w$, \eqref{eq:nonhomo} is equivalent to the equation:
\begin{equation}
    \label{eq:change_to_homogeneous}
    \left\{
        \begin{aligned}
            L\tilde{u}&=\tilde{f},x\in U,\\
            \tilde{u}&=0,x\in\partial U,\\
        \end{aligned}
    \right.
\end{equation}
where $\tilde{f}:=f-Lw\in H^{-1}(U)$.
\section{Existence of weak solutions}
\subsection{Lax-Milgram theorem}
We now introduce an abstract principle from linear functional analysis. 

Assume $H$ is a real Hilbert space, with norm $\|\;\|$ and inner product $(\;,\;)$. We let $\innerprod{\cdot}{\cdot}$ denote the pairing of $H$ with its dual space.
\begin{theorem}{Lax-Milgram Theorem}
    \label{thm:Lax-Milgram}
    Assume that 
    \begin{equation}
        \label{eq:bilinear_form_B}
        B:H\times H\rightarrow\mathbb{R}
    \end{equation}
    is a bilinear mapping, for which there exist constants $\alpha,\beta$ such that 
    \begin{equation}
        \label{eq:bounded}
        |B(u,v)|\le \alpha \norm{u}\norm{v}(u,v\in H)
    \end{equation}
    and 
    \begin{equation}
        \label{eq:elliptic}
        \beta\norm{u}^{2}\le B[u,u](u\in H).
    \end{equation}
    Finally, let $f:H\rightarrow R$ be a bounded linear functional on $H$.

    Then there exists a unique element $u\in H$ such that 
    \begin{equation}
        \label{eq:Lax-Milgram}
        B[u,v]=\innerprod{f}{v}
    \end{equation}
    for all $v\in H$.
\end{theorem}
\begin{remark}
    \begin{enumerate}
        \item If $B$ is symmetry, the condition \eqref{eq:bounded} and \eqref{eq:elliptic} means $B$ can derive an inner product on $H$.
        \item So, if $B$ is symmetry, theorem \ref{thm:Lax-Milgram} is a direct corollary of Riesz representation theorem.
        \item If $B$ isn't symmetry, Riesz representation theorem can transform $f\in H^{*}$ to $u_{f}\in H$, such that $\innerprod{f}{v}=(u_{f},v)$.
        \item So, we should show that $\forall u_{f}\in H$, $\exists u\in H$ such that $B[u,v]=(u_{f},v)$ for each $v\in H$.
    \end{enumerate}
\end{remark}
\begin{proof}

    By the remark above, we just need to show that $\forall u_{f}\in H$, $\exists u\in H$ such that $B[u,v]=(u_{f},v)$ for each $v\in H$.

    Consider an element $u\in H$. $B(u,v)$ is a bounded bilinear mapping, so the map 
    \begin{equation}
        \label{eq:Bu}
        B_{u}(v):=B[u,v]
    \end{equation}
    is a bounded linear functional. By Riesz representation theorem, $\exists !\tilde{u}\in H$ such that $B[u,v]=(\tilde{u},v)$ $\forall v\in H$. So we can define a map $A:H\rightarrow H$ maps $u$ to $\tilde{u}$.
    
    Then it's suffices to show that $A$ is a bounded linear isomorphism.

    First we should show that $A$ is linear. $\forall v\in H$, $\lambda_{1},\lambda_{2}\in\mathbb{R}$, $u_{1},u_{2}\in H$, we can see:
    \begin{equation}
        \label{eq:linear_A}
        \begin{aligned}
        (A(\lambda_{1}u_{1}+\lambda_{2}u_{2}),v)&=B[\lambda_{1}u_{1}+\lambda_{2}u_{2},v]\\
        &=\lambda_{1}B[u_{1},v]+\lambda_{2}B[u_{2},v]\\
        &=\lambda_{1}(Au_{1},v)+\lambda_{2}(Au_{2},v).
        \end{aligned}
    \end{equation}
    \eqref{eq:linear_A} shows that $A$ is a linear map. 

    Then we show that $A$ is bounded. As $B$ is bounded, we can see:
    \begin{equation}
        \label{eq:Bounded_A}
        \begin{aligned}
        \norm{Au}^{2}&=(Au,Au)\\
        &=B[u,Au]\\
        &\le \alpha \norm{u}\norm{Au}.
        \end{aligned}
    \end{equation}
    i.e. $\norm{Au}\le\alpha\norm{u}$. So $A$ is bounded.

    The next thing to do in the proof is to show $A$ is an injective. By \eqref{eq:elliptic}, we can see:
    \begin{equation}
        \label{eq:injective_A}
        \begin{aligned}
        \beta\norm{u}^{2}&\le B[u,u]\\
        &=(Au,u)\\
        &\le \norm{Au}\norm{u}.
        \end{aligned}
    \end{equation}
    i.e. $\beta\norm{u}\le\norm{Au}$. So $\beta\norm{u}\le \norm{Au}\le\alpha\norm{u}$, it means that $A$ is an injective. What's more, the range of $A$, marked as $R(A)$, is a closed set.
    
    Finally, we should show that $A$ is a surjective. If $R(A)\neq H$, since $R(A)$ is closed, there exists a nonzero element  $\omega\in H$ with $\omega\in R(A)^{\bot}$. Then: 
    \begin{equation}
        \beta\norm{\omega}^{2}\le B[\omega,\omega]=(A\omega,\omega)=0.
    \end{equation} 
    which means that $\norm{\omega}=0$, contradict! So $A$ is a surjective.

    This completes the proof of Lax-Milgram theorem.
\end{proof}

Lax-Milgram theorem gives an important method for us to analyze the existence of weak solution.
\subsection{Energy estimates and First Existence theorem}
Now, we return to the specific bilinear form $B[\;,\;]$ defined by \eqref{eq:bilinear_form_B1}, and try to use \textbf{Lax-Milgram theorem} to prove the first existence theorem.
\begin{theorem}{Energy estimates}
\label{thm:energy_estimate}
There exists constants $\alpha,\beta>0$ and $\gamma\ge 0$ such that 
\begin{equation}
    \label{eq:bounded_B}
    |B[u,v]|\le\alpha\norm{u}_{H_{0}^{1}(U)}\norm{v}_{H_{0}^{1}(U)}
\end{equation}
and 
\begin{equation}
    \label{eq:elliptic_B}
    \beta\norm{u}_{H_{0}^{1}(U)}^{2}\le B[u,u]+\gamma\norm{u}_{L^{2}(U)}^{2}.
\end{equation}
\end{theorem}
\begin{proof}
    First derive the inequality \eqref{eq:bounded_B}. According to \eqref{eq:bilinear_form_B1}, we can check that 
    \begin{equation}
        \label{eq:derive_energy_B}
        \begin{aligned}
            B[u,v]&\le\sum_{i,j=1}^{n}\int_{U}\norm{a^{ij}}_{L^{\infty}}|u_{x_{i}}||v_{x_{j}}|\dif x+\sum_{i=1}^{n}\int_{U}\norm{b^{i}}_{L^{\infty}}|u_{x_{i}}||v|\dif x+\int_{U}\norm{c}_{L^{\infty}}|u||v|\dif x\\
            &\le\sum_{i,j=1}^{n}\int_{U}\norm{a^{ij}}_{L^{\infty}}|Du||Dv|\dif x+\sum_{i=1}^{n}\int_{U}\norm{b^{i}}_{L^{\infty}}|Du||v|\dif x+\int_{U}\norm{c}_{L^{\infty}}|u||v|\dif x\\
            &\le\alpha\norm{u}_{H_{0}^{1}(U)}\norm{v}_{H_{0}^{1}(U)}.
        \end{aligned}
    \end{equation}
    Then, by the uniformly elliptic condition of coefficient matrix $a^{ij}(x)$, we can see:
    \begin{equation}
        \label{eq:elliptic_for_a}
        \begin{aligned}
        \theta\int_{U}|Du|^{2}\dif x&\le \int_{U}\sum_{i,j=1}^{n}a^{ij}(x)u_{x_{i}}u_{x_{j}}\dif x\\
        &=B[u,u]-\int_{U}\left(\sum_{i=1}^{n}b^{i}(x)u_{x_{i}}u\dif x+cu^{2}\right)\dif x\\
        &\le B[u,u]+\sum_{i=1}^{n}\int_{U}\norm{b^{i}(x)}_{\Linf}|u||Du|\dif x+c\int_{U}u^{2}\dif x\\ 
        \end{aligned}
    \end{equation}
    By Cauchy-Schwarz inequality with coefficient, we can see:
    \begin{equation}
        \label{eq:cauchy_schwarz}
        \int_{U}|u||Du|\dif x\le\epsilon\int_{U}|Du|^{2}\dif x+\frac{1}{4\epsilon}\int_{U}|u|^{2}\dif x.
    \end{equation}
    Choose $\epsilon$ such that $\epsilon\sum_{i=1}^{n}\norm{b^{i}}_{L^{\infty}}<\frac{\theta}{2}$, then exists constant $C>0$ such that 
    \begin{equation}
        \label{eq:inequality_for_ellip}
        \frac{\theta}{2}\int_{U}|Du|^{2}\dif x\le B[u,u]+C\int_{U}u^{2}\dif x.
    \end{equation}
    Finally, by Poincare-Friedrichs inequality, the equation \eqref{eq:elliptic_B} is true.
\end{proof}
By \eqref{eq:elliptic_B}, if $\gamma>0$, $B[u,v]$ isn't uniformly elliptic in general. So, $B[u,v]$ does't satisfy the hypotheses of Lax-Milgram theorem in general. So we should give some revisions on bilinear form $B$. Then, we derive the \textbf{first existence theorem for weak solutions.}
\begin{theorem}{First Existence Theorem for weak solutions}
    There is a number $\gamma\ge 0$ such that for each $\mu\ge\gamma$ and each function $f\in L^{2}(U)$, there exists a unique weak solution $u\in H_{0}^{1}(U)$ of the boundary-value problem 
    \begin{equation}
        \label{eq:modified_BVP}
        \left\{
            \begin{aligned}
                Lu+\mu u&=f,x\in U;\\
                u&=0,x\in\partial\Omega.\\
            \end{aligned}
        \right.
    \end{equation}
\end{theorem}
\begin{proof}
    Choose the parameter $\gamma$ as theorem \ref{thm:energy_estimate}, then define the bilinear form 
    \begin{equation}
        \label{eq:Bmu}
        B_{\mu}[u,v]:=B[u,v]+\mu(u,v).
    \end{equation}
    Then the weak form of equation \eqref{eq:modified_BVP} is 
    \begin{equation}
        B_{\mu}[u,v]=\innerprod{f}{v}.
    \end{equation}
    By \eqref{eq:elliptic_B}, as $\mu\ge\gamma$, the bilinear form $B_{\mu}$ satisfies uniformly elliptic condition. So by Lax-Milgram theorem, equation \eqref{eq:Bmu} has unique weak solution.
\end{proof}
\begin{remark}
    The first existence theorem for weak solutions is a milestone, but we still can't give the existence theorem of equation \eqref{eq:Elliptic_eq}. We should use Fredholm alternative theorem to derive the existence of weak solution.
\end{remark}
\subsection{Fredholm alternative and the solvability}
In this section, we show the Fredholm alternative theorem first, then employ this theorem to derive the existence theorem for weak solutions.

\begin{theorem}{Fredholm alternative}
    \label{thm:Fredholm}
    Let $K:H\rightarrow H$ be a compact linear operator, then:
    \begin{itemize}
        \item $\ker(I-K)$ is finite dimensional.
        \item $R(I-K)$ is closed.
        \item $R(I-K)=\ker(I-K^{*})^{\perp}$.
        \item $\ker(I-K)=\{0\}$ if and only if $R(I-K)=H$.
        \item $\dim\ker(I-K)=\dim\ker(I-K^{*})$.
    \end{itemize}
\end{theorem}
\begin{proof}
    Omitted.
\end{proof}
Then, derive the \textbf{dual problem} of equation \eqref{eq:Elliptic_eq}.
\begin{definition}{Dual problem}
    \begin{enumerate}
        \item The operator $L^{*}$, the formal adjoint of $L$, is:
        \begin{equation}
            \label{eq:dual_op}
            L^{*}v=-\sum_{i,j=1}^{n}\left(a^{ij}v_{x_{j}}\right)_{x_{i}}-\sum_{i=1}^{n}b^{i}v_{x_{i}}+(c-\sum_{i=1}^{n}b_{x_{i}}^{i})v,
        \end{equation}
        provided $b^{i}\in C^{1}(\bar{U})$.
        \item The adjoint bilinear form 
        \begin{equation}
            \label{eq:dualB}
            B^{*}:H_{0}^{1}(U)\times H_{0}^{1}(U)\rightarrow\mathbb{R}
        \end{equation}
        is defined by 
        \begin{equation}
            \label{eq:dual_B}
            B^{*}[v,u]=B[u,v]
        \end{equation}
        for all $u,v\in H_{0}^{1}(U)$.
        \item We say that $v\in H_{0}^{1}(U)$ is a weak solution of the adjoint problem 
        \begin{equation}
            \label{eq:adj_problem}
            \left\{
                \begin{aligned}
                    L^{*}v&=f,x\in U;\\
                    v&=0,x\in\partial U.
                \end{aligned}
            \right.
        \end{equation}
        provided $B^{*}[v,u]=(f,u)$ for all $u\in H_{0}^{1}(U)$.
    \end{enumerate}
\end{definition}
\begin{remark}
    \begin{itemize}
        \item \eqref{eq:dual_op} is called formal adjoint, for $(Lu,v)=(u,L^{*}v)$.
        \item \eqref{eq:adj_problem} is the dual form of equation \eqref{eq:Elliptic_eq}.
    \end{itemize}
\end{remark}