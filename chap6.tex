\chapter{Elliptic equation}
In this chapter, we will discuss on the general elliptic PDEs and its weak form. We will exploit two essentially distinct techniques, energy methods within Sobolev spaces and maximum principle methods.
\section{Definition}
\subsection{Elliptic equations}
We will in this chapter mostly study the boundary value problem(BVP):
\begin{equation}
    \label{eq:Elliptic_eq}
    \left\{
        \begin{aligned}
            Lu&=f\text{ in }U;\\
            u&=0\text{ on }\partial U,\\
        \end{aligned}
    \right.
\end{equation} 
where $U$ is an open, bounded subset of $\mathbb{R}^{n}$, and $u:\bar{U}\rightarrow\mathbb{R}$ is the unknown. Here $f$ is given, and $L$ denotes a second order differential operator have either the form:
\begin{equation}
    \label{eq:div-form_ellip}
    Lu=-\sum_{i,j=1}^{n}\left(a^{ij}(x)u_{x_{i}}\right)_{x_{j}}+\sum_{i=1}^{n}b^{i}(x)u_{x_{i}}+c(x)u
\end{equation}
or else 
\begin{equation}
    \label{eq:non_div-form_ellip}
    Lu=-\sum_{i,j=1}^{n}a^{ij}(x)u_{x_{i}x_{j}}+\sum_{i=1}^{n}b^{i}(x)u_{x_{i}}+c(x)u.
\end{equation}
The form \eqref{eq:div-form_ellip} is in \textbf{divergence form}, while the form \eqref{eq:non_div-form_ellip} is in \textbf{non-divergence form}. The requirement that $u=0$ on $\partial\Omega$ is called \textbf{Dirichlet's boundary condition}.

\begin{remark}
    \begin{itemize}
        \item Different with Evans chapter 2, $u\in C^{2}(U)$ is unnecessary. So we should discuss on the weak form of equation \eqref{eq:Elliptic_eq}.
        \item If $a^{ij}\in C^{2}(U)$, the form \eqref{eq:div-form_ellip} and \eqref{eq:non_div-form_ellip} are equivalent in general. But the form \eqref{eq:div-form_ellip} is more natural for energy methods, and the form \eqref{eq:non_div-form_ellip} is more appropriate for maximum principle techniques.
        \item Assumption: \textbf{symmetry condition}
        \begin{equation*}
            \label{eq:symmetry_cond}
            a^{ij}=a^{ji}.
        \end{equation*}
    \end{itemize}
\end{remark}
Now, give an important property of the differential operator $L$.
\begin{definition}
    \label{defn:uniformly_elliptic}
    We say the partial differential operator $L$ is (uniformly) elliptic if there exists a constant $\theta>0$ such that
    \begin{equation}
        \sum_{i,j=1}^{n}a^{ij}(x)\xi_{i}\xi_{j}\ge\theta |\xi|^{2}
    \end{equation}
    for a.e. $x\in U$ and all $\xi\in\mathbb{R}^{n}$.
\end{definition}

\begin{remark}
    \begin{itemize}
        \item $L$ is uniformly elliptic means the matrix $(a^{ij}(x))$ is positive definite $\forall x\in U$.
        \item The converse proposition of the above proposition isn't true.
        \item Special case: $a^{ij}(x)\equiv\delta_{ij}, b^{i}\equiv 0, c^{i}\equiv 0$. In this case, the equation \eqref{eq:Elliptic_eq} is the \textbf{Poisson equation}.
    \end{itemize}
\end{remark}
\subsection{Weak solutions}
\textbf{Motivation:}In general case, we can only assume that 
\begin{equation}
    \label{eq:assumption_for_ellip}
    a^{ij},b^{i},c\in L^{\infty}(U),f\in L^{2}(U).
\end{equation} 
In this case, maybe we can't find $u\in C^{2}(U)$ such that $u$ satisfies \eqref{eq:Elliptic_eq}, so we try to derive the \textbf{weak form} of the solution $u$, such that $u\in H_{0}^{1}(U)$. In practise, choose \textbf{test function} $v\in C_{c}^{\infty}(U)$, then define the linear functional 
\begin{equation}
    \label{eq:functional_of_u}
    u^{*}(v):=\int_{U}uv\dif x.
\end{equation}
Then consider the dual form of \eqref{eq:Elliptic_eq}, i.e. 
\begin{equation}
    \label{eq:dual_form}
    (Lu)^{*}(v)=f^{*}(v)\text{ }\forall v\in C_{c}^{\infty}(U).
\end{equation}
It's just the \textbf{weak form} of equation \eqref{eq:Elliptic_eq}. In this section, we choose the \textbf{divergence form} of operator $L$.
\begin{theorem}
    The weak form of equation \eqref{eq:Elliptic_eq} is 
    \begin{equation}
        \label{eq:weak_form}
        \int_{U}\sum_{i,j=1}^{n}a^{ij}u_{x_{i}}v_{x_{j}}+\sum_{i=1}^{n}b^{i}u_{x_{i}}v+cuv\dif x=\int_{U}fv\dif x.
    \end{equation}
\end{theorem}
\begin{proof}
    It's suffices to derive the expression of $\int_{U}vLu\dif x$. By Gauss-Green formula, for a vector field $F\in C^{1}(\mathbb{R}^{n})$, we can see:
    \begin{equation}
        \label{eq:Gauss-Green}
        \int_{U}v\nabla\cdot F\dif x=\int_{\partial U}v F\cdot n\dif S(x)-\int_{U}F\cdot\nabla v\dif x.
    \end{equation}
    Choose the vector field $F=\begin{bmatrix}
        \sum a^{i1}(x)u_{x_{i}}\\
        \sum a^{i2}(x)u_{x_{i}}\\
        \vdots\\
        \sum a^{in}(x)u_{x_{i}}\\
    \end{bmatrix}$, as $v=0$ on $\partial \Omega$, by \eqref{eq:Gauss-Green}, we can see:
    \begin{equation}
        \begin{aligned}
            \int_{U}\sum_{i,j=1}^{n}\left(a^{ij}(x)u_{x_{i}}\right)_{x_{j}}v\dif x&=\int_{U}v\nabla\cdot F\dif x\\
            &=-\int_{U}F\cdot\nabla v\dif x\\
            &=-\int_{U}\sum_{i,j=1}^{n}a^{ij}u_{x_{i}}v_{x_{j}}\dif x.\\
        \end{aligned}
    \end{equation}
    Then:
    \begin{equation}
        \label{eq:weak}
        \int_{U}vLu\dif x=\int_{U}\sum_{i,j=1}^{n}a^{ij}u_{x_{i}}v_{x_{j}}+\sum_{i=1}^{n}b^{i}u_{x_{i}}v+cuv\dif x=\int_{U}vf\dif x=f^{*}(v).
    \end{equation}
\end{proof}
By \eqref{eq:weak_form}, we can derive the following definitions.
\begin{definition}
    \begin{enumerate}
        \item The bilinear form $B[\;,\;]$ associated with the divergence form elliptic operator defined by \eqref{eq:div-form_ellip} is:
        \begin{equation}
            \label{eq:bilinear_form_B}
            B[u,v]:=\int_{U}\sum_{i,j=1}^{n}a^{ij}u_{x_{i}}v_{x_{j}}+\sum_{i=1}^{n}b^{i}u_{x_{i}}v
        \end{equation}
        for $u,v\in H_{0}^{1}(U)$.
        \item We say that $u\in H_{0}^{1}(U)$ is a weak solution of the BVP \eqref{eq:Elliptic_eq} if 
        \begin{equation}
            B[u,v]=(f,v)
        \end{equation}
        for all $v\in H_{0}^{1}(U)$, where $(\;,\;)$ denotes the inner product in $L^{2}(U)$.
    \end{enumerate}
\end{definition}

More generally, let us consider the BVP 
\begin{equation}
    \label{eq:revised_BVP}
    \left\{
        \begin{aligned}
            Lu&=f^{0}-\sum_{i=1}^{n}f_{x_{i}}^{i},x\in U,\\
            u&=0,x\in\partial U,
        \end{aligned}
    \right.
\end{equation}
where $f^{i}\in L^{2}(U)$. Then we say $u\in H_{0}^{1}(U)$ is a weak solution of problem \eqref{eq:revised_BVP} if 
\begin{equation}
    B[u,v]=\innerprod{f}{v}
\end{equation}
for all $v\in H_{0}^{1}(U)$, where $\innerprod{\cdot}{\cdot}$ is the pairing of $H^{-1}(U)$ and $H_{0}^{1}(U)$.

For non-homogeneous elliptic PDE, i.e.
\begin{equation}
    \label{eq:nonhomo}
    \left\{
        \begin{aligned}
            Lu&=f,x\in U,\\
            u&=g,x\in \partial U.\\
        \end{aligned}
    \right.
\end{equation}
By trace theorem, $\exists w\in H^{1}(U)$ such that the trace of $w$ is $g$. Then define $\tilde{u}:=u-w$, \eqref{eq:nonhomo} is equivalent to the equation:
\begin{equation}
    \label{eq:change_to_homogeneous}
    \left\{
        \begin{aligned}
            L\tilde{u}&=\tilde{f},x\in U,\\
            \tilde{u}&=0,x\in\partial U,\\
        \end{aligned}
    \right.
\end{equation}
where $\tilde{f}:=f-Lw\in H^{-1}(U)$.